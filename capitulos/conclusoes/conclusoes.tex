\chapter{Conclusão}



\section{Conclusão do Trablho}

Este trabalho teve como objetivo a implantação e cobertura de testes automatizados durante o desenvolvimento de uma aplicação que foi desenvolvida com tecnologia de classificação de sinais de áudio para dar suporte a pessoas com deficiência auditiva ou surda.

Foi possível observar que a implementação da automação dos testes desde do começo do desenvolvimento e em conjunto com o desenvolvedor (time) pode impactar no desenho da arquitetura e na melhor aplicação das ferramentas e estruturação do projeto. Ainda em tempo de desenvolvimento das funcionalidades foi possível corrigir os erros que eram encontrados durante a implantação da automação dos testes.

Sendo assim, aplicação de testes automatizados em todos os níveis de testes, como a pirâmide ideal de testes indica, foi empregado através de técnicas e ferramentas que deram o suporte necessário desta estratégia de implementação. Um dos benefícios percebidos foram os ganhos de tempo na execução dos testes automatizados e a frequência com eles podem ser executados, uma vezes que eles foram desenvolvidos, podem ser reproduzidos em vários ambientes e ao longo de alterações para validar novas versões a um baixo custo. Tal característica ressalta o potencial dos testes automatizados para testes de regressão.

Por fim,  é perceptível que a implantação de testes automatizados não é um processo fácil, que demanda uma mudança, tanto de como o processo de testes são executados quanto na capacitação  técnica dos responsáveis pela automatização dos testes do time, mas que sua utilização sistemática pode ser eficaz e trazer benefícios reais a um projeto de desenvolvimento de software. 


\section{Trabalhos futuros}

Em conjunto com o desenvolvimento do trabalho de conclusão de curso que foi realizado em paralelo com este, chegou se a conclusão que é factível dar continuidade ao desenvolvimento e aprimoramento da aplicação juntamente com a automação dos testes. Realizando melhorias nos próximos passos do desenvolvimento e aprimorando as técnicas e aderindo novas tecnologias para validar o quanto antes a aplicação. Ainda também garantir testes a nível final com o usuário e de interface, coletando novos dados para continuar aprofundando estudos deste tipo de estratégia.
