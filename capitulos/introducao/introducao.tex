\chapter{Introdução}

\section{Apresentação}

O Manifesto Ágil \cite{agileManifesto}, criado em 2001, revolucionou o mercado de Tecnologia da Informação, pois, até então, a maneira de se desenvolver software mostrou-se falha em diversos aspectos. Contudo, a transição entre os modelos antigos (conhecidos como “plan-driven”) e o moderno (conhecido como Ágil) é uma tarefa não-trivial.

As principais motivações por trás da adoção ágil de desenvolvimento de software são: a melhoria da qualidade do produto final, aumento do ânimo dos desenvolvedores e satisfação do cliente. Entretanto, adoção ágil sempre vem com desafios especiais e, para que ela ocorra com sucesso, mudanças fundamentais na cultura da organização são necessárias \cite{Hajjdiab2011}. Esse projeto visa expor lições aprendidas por empresas de desenvolvimento de software que passaram por esse processo, a fim de apoiar a transição de organizações que ainda têm receio de migrar para Ágil.

\section{Justificativas}

Muitas abordagens diferentes podem ser aplicadas a desenvolvimento de software \cite{Kettunen2010}, cada uma com suas peculiaridades. De acordo com Shore e Warden \cite{Shore2007}, metodologias denominadas ágeis tornaram-se bastante populares, sendo utilizadas por grandes corporações como: Google, Yahoo!, Symantec, Microsoft, etc. Entretanto, não se deve simplesmente aplicá-las porque são populares. Existem vários estudos na literatura que mostram casos falhos de adoção de métodos ágeis \cite{Krasteva2008}. É preciso entender a fundo essas novas metodologias, analisar seus pontos positivos e negativos e, principalmente, refletir se elas agregariam valor ao produto final a ser entregue. A maneira menos dolorosa de adquirir esse conhecimento é através do compartilhamento de lições aprendidas.

Sendo assim, é de suma importância a coleta de experiências para que se possa aprender sobre essa transição com aqueles que já passaram por esse processo e que, além disso, querem contribuir com a comunidade de desenvolvimento de software.

\section{Objetivos} 

O objetivo geral desse trabalho é disponibilizar para acesso da indústria e academia de software um conjunto de lições aprendidas mais relatadas com a adoção de desenvolvimento ágil por organizações de software.

Com ênfase no alcance do objetivo geral, o trabalho definiu os seguintes objetivos específicos:

\begin{itemize}
	\item Investigar, através de pesquisa exploratória, a necessidade e contribuição de trabalhos dessa natureza.
	\item Adquirir conhecimento sobre fundamentos teóricos relacionados como desenvolvimento ágil de software e lean.
	\item Realizar uma revisão utilizando como base a metodologia de revisão sistemática \cite{Barbara04} para coletar lições aprendidas provenientes de trabalhos publicados pela comunidade científica.
	\item Realizar uma pesquisa exploratória para coletar lições aprendidas provenientes de trabalhos publicados pela indústria de software nas principais conferências nesse contexto.
	\item Identificar um conjunto de benefícios e desafios representativos a partir dos resultados das pesquisas realizadas.
	\item Validar o trabalho realizado através de uma metodologia de survey \cite{Babbie1990}, aplicando um questionário com especialistas brasileiros reconhecidos na área de metodologias ágeis.
	\item Gerar um protótipo de um banco de dados de lições aprendidas para acesso livre pela comunidade de software.
\end{itemize}

\section{Contribuições obtidas}

Com esse trabalho, foi possível criar uma fonte de informações relacionadas a processos de adoção ágil em empresas de desenvolvimento de software. Além disso, foi gerado um conteúdo relevante para a criação de um protótipo de banco de dados de lições aprendidas que, dependendo do interesse da comunidade, pode vir a se tornar uma ferramenta colaborativa para apoio à institucionalização do desenvolvimento ágil em organizações de software.

\section{Organização do trabalho}

Esse trabalho é composto por mais quatro capítulos. No capítulo 2 está exposto o método utilizado para revisar a literatura, juntamente com o conjunto de artigos científicos e relatos de experiência resultantes da revisão realizada. No capítulo 3 encontra-se a análise e a categorização das lições aprendidas coletadas. O capítulo 4 apresenta a validação dos resultados encontrados, através da aplicação de um questionário. No capítulo 5 constam a conclusão da pesquisa e possíveis trabalhos futuros.
