\chapter{Metodologia}

Tendo em vista que o teste de software é uma das principais atividades da Engenharia 
de Software e que demanda um grande esforço em projetos de desenvolvimento, esse 
trabalho de conclusão de curso parte do seguinte problema: como implantar uma 
solução de testes automatizados em um projeto de software, de forma que a 
cobertura dos testes esteja presente em todos os níveis do sistema, durante todo o 
ciclo de vida do desenvolvimento? E quais podem ser o impacto dessa estratégia?

Para responder esta pergunta, o presente trabalho teve como foco realizar o  desenvolvimento da aplicação juntamente com o desenvolvedor,  trabalhando na qualidade do código gerado desde da concepção da ideia da aplicação e tendo conhecimento de toda a arquitetura do desenvolvimento afim de entender, selecionar e implantar a automação dos testes que foram necessários para validação final.

Estão fora do escopo deste trabalho o entendimento e análise profunda do framework de classificação de áudio jMir como também a inteligência na classificação de áudio por trás dos algoritmos que é utilizado.

Como parte desta proposta foram efetuadas várias atividades, a seguir serão apresentados os passos para execução da realização deste trabalho:

\section{Definição do Problema}

Com base em conhecimento prévio na área de testes em times ágeis,  foi observado todo o esforço e importância nas atividades envolvendo qualidade de software.  Lidando no dia-a-dia com impasses de falta de tempo para realizar testes manuais, a não cobertura de testes durante o desenvolvimento, a exclusão da qualidade do software durante o ciclo de vida do desenvolvimento e muitos outros pontos foram inputs para a motivação da pesquisa e implantação de automação dos testes durante o desenvolvimento e o impacto que isso poderia geral no resultado final do produto.

\section{Definição da Solução para aplicação da automação dos testes}

\begin{itemize}
	\item Pesquisa sobre tecnologias de áudio
	
Por motivações pessoais para o desenvolvimento de uma aplicação que envolvesse tecnologias que tratam e possibilitam a manipulação de áudio, uma pesquisa sobre áudio monitoramento foi realizada.  Com base nesta pesquisa, a proposta é utilizar um pacote de software que possibilita a classificação de sinais sonoros. 

	\item Identificação da Solução de um problema	
	
Ao encontrar ferramentas e apoio de tecnologias que consigam prover a classificação de sinais de áudio,  a realização de um Brainstorm foi feita para coletar ideias que serviram de base para a definição de um problema e criação de uma aplicação.

Brainstorm (ou Brainstorming) é um processo de geração de ideias que explora a capacidade criativa das pessoas,  por meio da produção em massa de ideias em um curto espaço de tempo, para posterior avaliação.  Foram escolhidos alguns temas para facilitar o encaminhamento de problemas envolvendo a classificação de áudio, tais como: Saúde, educação, segurança, música, trânsito, redes sociais(comunicação). A fim de melhorar a identificação das áreas mencionadas, cada área possuía uma cor em um cartão. Cada rodada durava 3 minutos, para que os participantes gerassem ideias correspondentes a área selecionada, ao fim deste tempo, a área em destaque era trocada e repetia os mesmo passos. Depois da inserção de ideias, foram lidas todas as sugestões e melhor explicadas aos participantes. Dando inicio a próxima fase que foi votar nas três melhores ideias geradas. Ao final da votação foi escolhido as três ideais mais votadas, o grupo então precisou pensar quais problemas cada ideia poderia resolver. 
	
	\item Concepção da aplicação/solução
	
Com base na ideia gerada através do Brainstorm mencionado, foi concebida a estória do usuário principal para a definição do escopo da aplicação.
 
\end{itemize}

\section{Referencial Teórico}

Definida a aplicação em que esta proposta seria aplicada, foi realizada a revisão da literatura com intuito de sintetizar os principais conceitos, definições e técnicas. Os assuntos que serviram de tema para este estudo teórico foram: Testes de Software, Tipos de Testes de Software, Níveis de Testes de Software e Automação de Testes.

\section{Seleção dos tipos de testes}

Com o escopo da aplicação definida, com o conhecimento necessário sobre quais tecnologias de classificação de áudio utilizar e com o embasamento teórico como base, foi possível realizar o reconhecimento de quais tipos de testes poderiam ser aplicados, quais tecnologias utilizar para cada tipo de testes selecionado. 

\section{Aplicação de testes automatizados ao longo do desenvolvimento}

Como parte da proposta deste trabalho, testes eram automatizado com base no código que era desenvolvido ao mesmo tempo, gerando validação do que era produzido, reparação de falhas e divergências da implementação. O desenvolvimento do sistema mencionado, foi feito em paralelo como trabalho de conclusão de curso do aluno Luan Reis.

\section{Analise do resultado dos testes automatizados}

Com base nos resultados da aplicação da automação dos testes durante o desenvolvimento do começo ao fim da aplicação desenvolvida, foi possível levantar vários pontos como benefícios e reconhecer o impacto de tal estratégia de testes. Por fim, tornou-se concreto garantir que o sistema desenvolvido tinha todos os níveis de testes cobertos pela automação dos testes. 

\section{Conclusão}

Nesta fase é realizada uma reflexão sobre viabilidade de implantar testes automatizados como parte do processo de desenvolvimento e de como isso pode trazer benefícios a qualidade final do produto.  Onde trabalhos futuros podem ser gerados através da aplicação e melhoria de estratégias como esta. 