\chapter{Referêncial Teórico}


É apresentado nesse capítulo um conjunto de definições relevantes para o tema dessa proposta de trabalho. Foi realizada uma pesquisa sistemática para selecionar os principais tópicos apoiam o entendimento deste trabalho. Os detalhes estão descritos nas próximas seções.

\section{Teste de Software}

Teste de software é o processo que realiza a avaliação do sistema ou de seus componentes com o objetivo de desvendar o comportamento para garantir que os seus requisitos estão de acordo com o esperado ou não. Em outras palavras teste é execução do sistema, a fim de identificar eventuais lacunas, erros ou requisitos que não foram implementados de acordo com as necessidades requeridas \cite{tutorialsPoint}. Segundo \cite{ansiieee1059}, Teste pode ser definido como o processo de análise do software para detectar as diferenças entre condições existentes e necessárias e avaliar as características do software.

Um bom teste pode ser aquele que tem alta probabilidade de encontrar erros que ainda não foram expostos e um teste bem sucedido é aquele que revela um erro ainda não-descoberto. O teste pode ser manual, automatizado, ou ainda a combinação de ambos \cite{Pressman2002}. A redução de custo, tempo e retrabalho são proporcionais ao quão cedo o processo de testes for iniciado \cite{tutorialsPoint}.

\section{Tipos de Teste de software}

Testes de software podem ser divididos em duas categorias, testes manuais e automatizados.

\subsection{Testes Manuais}

		Este tipo de teste é feito pela execução manual do software, ou seja, sem o uso de qualquer ferramenta automatizada ou qualquer script. O testador assume o papel do usuário final para realizar os testes e identificar qualquer comportamento que não seja esperado ou revelar defeitos. Costumam usar planos de teste, casos de teste e/ou cenários para garantir a integridade dos testes. Também pode ser incluso dentro deste tipo os testes exploratórios, no qual o testador irá explorar o software com o intuito de encontrar erros\cite{tutorialsPoint}.
		
\subsection{Testes Automatizados}

Testes automatizados podem ser definidos como automação de atividades de teste de software, incluindo o desenvolvimento e execução de scripts de testes, verificação de requisitos, como também a utilização de ferramentas de teste automatizadas \cite{Dustin1999}.
 
Entre as razões para o uso da automação dos testes de software podemos citar, por exemplo, executar testes manuais é mais demorado, uso da automação de testes aumenta a  eficiência processo, em um fatia particular temos os testes de regressão, onde os casos de testes são executados de forma iterativa, depois de alterações no software \cite{Dustin1999}.
		
\section{Níveis de testes de Software}

Em \cite{Pressman2002} aponta que a realização de testes apenas quando o sistema está construído é uma abordagem ineficaz. Segundo o autor, a estratégia de testes de software de possuir uma abordagem incremental, começando com os testes de unidades, seguindo com os testes de integração, culminando com os testes do sistema final e ainda acoplando testes que se enquadrem em testes de aceitação.

\subsection{Testes Funcionais}

Este tipo de teste é baseado nas especificações do software que será testado, a aplicação será testada através do fornecimento de entradas e em seguida, os resultados serão examinados para garantir a conformidade com os requisitos de tal funcionalidade \cite{tutorialsPoint}. Basicamente testar se os componentes e o sistema estão feitos, uma atividade ou uma função especifica do código está coerente com o esperado. Neste nível de teste perguntas como "O usuário poderá fazer isso?" ou "Esta função em particular funciona?" podem ser validadas tipicamente através de especificações de requisitos ou funcional \cite{Leung1990}. 

\subsection{Testes Unitários}

É a menor parte que pode ser testada do código que compõe o software, são de granulação fina, comportamento extremamente rápido, por exemplo, métodos de uma classe, uma classe ou classes que podem estar relacionadas. Não verificam o comportamento de unidades integradas com outros serviços ou dependências, garantem que sua função como unidade estão funcionando. Seu objetivo é isolar parte do programa e mostrar que partes individuais estão corretas em termos de requisitos e funcionalidades \cite{James2012}.  Existe um limite para cenários e dados que podem ser aplicados ao nível de testes de unidade, quando é nítido esse esgotamento é necessário o mistura com outras unidades do código e diferentes tipos de testes \cite{XiangFeng2011}.

\subsection{Testes de Integração}

Neste caso, é realizada a combinação de partes da aplicação que serão agrupadas para determinar se funcionamento desse conjunto está trabalhando corretamente \cite{XiangFeng2011}. Alguns defeitos não podem ser encontrados a nível unitário, mas são revelados através da integração com núcleos específicos \cite{Pachawan2014}.




\subsubsection{Definição da pergunta}

O primeiro passo foi definir a questão a ser respondida ao final da pesquisa: ``\textit{Quais as principais lições aprendidas com a adoção de metodologias ágeis por parte de empresas de desenvolvimento de software na atualidade?}".

\subsubsection{Bases de dados relevantes}

Após a definição da pergunta a ser respondida, o próximo passo foi definir as fontes de informação mais confiáveis e relevantes para o tema (Tabela \ref{tab:basesDeDados}).

\begin{table}[H]
	\centering
	\begin{tabular}{| l | r |} \hline \textbf{Nome} & \textbf{Referência} \\ \hline
		CAPES & http://www.capes.gov.br/ \\ \hline
		ACM & http://www.acm.org/ \\ \hline
		IEEE & http://ieeexplore.ieee.org/ \\ \hline
		Google Scholar & http://scholar.google.com/ \\ \hline
		Springer Link & http://link.springer.com/ \\ \hline
	\end{tabular}
	\caption{Bases de dados consultadas na pesquisa}
	\label{tab:basesDeDados}
\end{table}

\nomenclature{CAPES}{Coordenação de Aperfeiçoamento de Pessoal de Nível Superior}%
\nomenclature{ACM}{Association for Computing Machinery}%
\nomenclature{IEEE}{Instituto de Engenheiros Eletricistas e Eletrônicos}%

\subsubsection{Palavras-chave}

Nesse passo, foram definidas as palavras-chave (Tabela \ref{tab:palavrasChave}) buscadas nas bases de dados selecionadas.

\begin{table}[H]
	\centering
	\begin{tabular}{| l |} \hline \textbf{Palavras-chave} \\ \hline
		Agile Adoption \\ \hline
	\end{tabular}
	\caption{Conjunto de palavras-chave utilizadas na pesquisa}
	\label{tab:palavrasChave}
\end{table}

\subsubsection{Critérios de exclusão}

\begin{table}[H]
	\centering
	\begin{tabular}{| c | r |} \hline \textbf{Base de dados}  & \textbf{Quantidade} \\ \hline
		ACM & 106 \\ \hline
		CAPES & 51 \\ \hline
		Google Scholar & 849 \\ \hline
		IEEE & 30 \\ \hline
		Springer Link & 6 \\
		\hline
	\end{tabular}
	\captionsetup{justification=centering}
	\caption{Quantidade de material encontrado antes da aplicação dos critérios de exclusão}
	\label{tab:quantidadeDeMateriaisAntes}
\end{table}

A base da literatura foi selecionada através do critério de exclusão de artigos. Como existem diversos materiais publicados na área de adoção ágil nas bases de dados selecionadas, foram encontrados cerca de 1000 artigos utilizando-se o conjunto de palavras-chave \textit{``Agile Adoption"} (Tabela \ref{tab:quantidadeDeMateriaisAntes}). Porém, muitos dos dados coletados nessa pesquisa preliminar eram antigos e poderiam não fazer mais muito sentido nos dias atuais. Sendo assim, foi adicionado um segundo critério de exclusão: a data de publicação. Apenas materiais atuais (entre 2010 e 2013) seriam analisados. O contingente de trabalhos passou para, em média, 550 artigos.

Outro ponto importante levado em consideração foi a relevância do tema nos trabalhos restantes. O filtro aplicado foi o de se encontrar as palavras-chave no título ou resumo. Com isso, o número de trabalhos reduziu bastante, para cerca de 50 artigos. Um ponto que vale a pena ser mencionado é que, nessa etapa do processo, caso o critério fosse levado extremamente à risca, bons materiais seriam eliminados. Esse fator foi analisado na etapa posterior.

O último critério de exclusão utilizado nessa pesquisa foi uma análise qualitativa do material restante. Após uma revisão, artigo por artigo, 16 foram escolhidos, estando 2 deles fora do critério de exclusão anterior, mas com conteúdo relevante para a pesquisa. A Tabela  \ref{tab:quantidadeDeMateriais} mostra com detalhes a quantidade de materiais selecionados em cada etapa.

\begin{table}[H]
	\centering
	\begin{tabular}{| c | l | r |} \hline \textbf{Critério de exclusão} & \textbf{Base de dados}  & \textbf{Quantidade} \\ \hline
		\multirow{4}{*}{Artigos entre 2010 e 2013} 
			& ACM & 26 \\ \cline{2-3}
			& CAPES & 37 \\ \cline{2-3}
			& Google Scholar & 459 \\ \cline{2-3}
			& IEEE & 16 \\ \cline{2-3}
			& Springer Link & 5 \\ \cline{2-3}
		\hline \hline
		\multirow{4}{*}{Palavras-chave no título e/ou resumo} 
			& ACM & 1 \\ \cline{2-3}
			& CAPES & 2 \\ \cline{2-3}
			& Google Scholar & 38 \\ \cline{2-3}
			& IEEE & 14 \\ \cline{2-3}
			& Springer Link & 0* \\ \cline{2-3}
		\hline \hline
		\multirow{4}{*}{Análise crítica}
			& ACM & 0 \\ \cline{2-3}
			& CAPES & 0 \\ \cline{2-3}
			& Google Scholar & 4 \\ \cline{2-3}
			& IEEE & 10 \\ \cline{2-3}
			& Springer Link & 2 \\ \cline{2-3}
		\hline
	\end{tabular}
	\captionsetup{justification=centering}
	\caption{Quantidade de material encontrado em cada passo da pesquisa}
	\label{tab:quantidadeDeMateriais}
\end{table}

\subsection{Artigos selecionados}

O produto final dessa revisão sistemática foi um conjunto de trabalhos acadêmicos relevantes à pesquisa (Tabelas \ref{tab:artigosIEEE}, \ref{tab:artigosGoogle} e \ref{tab:artigosSpringer}).

\begin{table}
	\centering
	\captionsetup{justification=centering,margin=1cm}
	\begin{tabularx}{\linewidth}{ | X | p{3cm} | } \hline \multicolumn{2}{|c|}{\textbf{Base de dados: IEEE}} \\ \hline
	\multicolumn{1}{|c|}{\textbf{Título do artigo}} & \multicolumn{1}{|c|}{\textbf{Referência}} \\ \hline
		Agile Adoption Experience: A Case Study in the U.A.E & \cite{Hajjdiab2011} \\ \hline
		Evolving to Agile: A story of agile adoption at a small SaaS company & \cite{Block2011} \\ \hline
		Factor Analysis: Investigating Important Aspects for Agile Adoption in Malaysia & \cite{Asnawi2012} \\ \hline
		Adobe Premiere Pro Scrum Adoption: How an agile approach enabled success in a hyper-competitive landscape & \cite{Adobe2012} \\ \hline
		Scaling Agile Methods to Regulated Environments: An Industry Case Study & \cite{Fitzgerald2013} \\ \hline
		The Maturation of Agile Software Development Principles and Practice: Observations on Successive Industrial Studies in 2010 and 2012 & \cite{Bustard2013} \\ \hline
		Have Agile Techniques been the Silver Bullet for Software Development at Microsoft? & \cite{Microsoft2013} \\ \hline
		The Agile Office: Experience Report from Cisco’s Unified Communications Business Unit & \cite{Cisco2011} \\ \hline
		The impact of agile principles and practices on large-scale software development projects A multiple-case study of two projects at Ericsson & \cite{Ericsson2013} \\ \hline
		Evaluating the Effect of Agile Methods on Software Defect Data and Defect Reporting Practices & \cite{Korhonen2010} \\ \hline
	\caption{Artigos científicos utilizados na pesquisa encontrados na base de dados IEEE}
	\label{tab:artigosIEEE}
	\end{tabularx}
\end{table}

\begin{table}[H]
	\centering
	\captionsetup{justification=centering,margin=1cm}
	\begin{tabularx}{\linewidth}{ | X | p{3cm} | } \hline \multicolumn{2}{|c|}{\textbf{Base de dados: Google Scholar}} \\ \hline
	\multicolumn{1}{|c|}{\textbf{Título do artigo}} & \multicolumn{1}{|c|}{\textbf{Referência}} \\ \hline
		DoD Agile Adoption: Necessary Considerations, Concerns, and Changes & \cite{Lapham2012} \\ \hline
		Software Tools and Processes: A Key Factor in Successful Agile Adoption & \cite{Arikpo2011} \\ \hline
		Inclusion of e-Assist to increase Agile Adoption & \cite{Radha2012} \\ \hline
		Agile Adoption Story from NHN & \cite{Eunha2012} \\ \hline
	\caption{Artigos científicos utilizados na pesquisa encontrados na base de dados Google Scholar}
	\label{tab:artigosGoogle}
	\end{tabularx}
\end{table}

\begin{table}[H]
	\centering
	\captionsetup{justification=centering,margin=1cm}
	\begin{tabularx}{\linewidth}{ | X | p{3cm} | } \hline \multicolumn{2}{|c|}{\textbf{Base de dados: Springer Link}} \\ \hline
	\multicolumn{1}{|c|}{\textbf{Título do artigo}} & \multicolumn{1}{|c|}{\textbf{Referência}} \\ \hline
		The evolution of agile software development in Brazil: Education, research, and the state-of-the-practice & \cite{Claudia2013} \\ \hline
		Evaluating the impact of an agile transformation: a longitudinal case study in a distributed  & \cite{Nokia2013} \\ \hline
	\caption{Artigos científicos utilizados na pesquisa encontrados na base de dados Springer Link}
	\label{tab:artigosSpringer}
	\end{tabularx}
\end{table}

O artigo \cite{Hajjdiab2011} apresentou um estudo de caso relatando o processo de adoção de métodos ágeis em uma entidade do governo dos Emirados Árabes Unidos. O relato \cite{Block2011} contou a história de adoção ágil de uma pequena empresa especializada em SaaS. O estudo \cite{Asnawi2012} focou na identificação de importantes aspectos na adoção ágil de empresas de desenvolvimento de software da Malásia. O artigo \cite{Adobe2012} descreveu como o pensamento ágil tem ajudado o Premiere Pro a ter ganhos significativos de vendas, qualidade e sustentabilidade. O estudo \cite{Fitzgerald2013} identificou pontos de tensão em adoção ágil e os ilustrou através de um detalhado caso em que uma abordagem ágil foi implementada com sucesso em um ambiente controlado. A proposta de \cite{Bustard2013} foi descrever o resultado de uma pesquisa feita na indústria cujo foco era entender a maturidade do modelo ágil no norte da Irlanda. A pesquisa \cite{Microsoft2013} rastreou mudanças de atitude e técnicas do processo de adoção ágil na Microsoft através de uma grande pesquisa longitudinal durante um período de seis anos. O artigo \cite{Cisco2011} relatou uma experiência da Cisco ao estabelecer um escritório ágil, descrevendo seu background histórico, seu modelo de governança, onde ele se encaixava na estrutura da organização, atividades primárias, desafios encontrados, etc. A proposta da pesquisa \cite{Ericsson2013} feita na Ericsson foi contribuir com evidências empíricas do impacto do uso de princípios e práticas ágeis na indústria em projetos de larga-escala. O estudo \cite{Korhonen2010} analisou dados de defeitos encontrados num projeto distribuído de uma empresa nos seus doze primeiros meses de transformação ágil. O trabalho \cite{Lapham2012} relatou um caso de um processo de adoção ágil feito no Departamento de Defesa dos Estados Unidos. O artigo \cite{Arikpo2011} teve como foco principal discutir as demandas de gerências de requisito, projeto e configuração, além de processos e ferramentas que facilitaram o processo de adoção ágil. O estudo \cite{Radha2012} discutiu os riscos envolvidos se pessoas não-capacitadas são escolhidas para desenvolver utilizando ágil, bem como maneiras de mitigá-los. O trabalho \cite{Eunha2012} mostrou como começar o processo de transição para desenvolvimento ágil através de dicas, tutoriais orientando como mover times de desenvolvimento de cascata para ágil, análise de benefícios e de possíveis problemas. No artigo \cite{Claudia2013} foi apresentada uma visão global da evolução do movimento ágil no Brasil, seus principais defensores na academia e na indústria, iniciativas educacionais e uma pesquisa que mostrou o estado da arte da indústria de TI no Brasil. O relato \cite{Nokia2013} foi baseado nos resultados de um estudo de caso que analisou o impacto que a introdução de práticas ágeis fez em uma grande companhia que faz parte da Nokia Siemens Networks.

\nomenclature{SaaS}{Software as a Service}%
\nomenclature{TI}{Tecnologia da Informação}%

\section{Relatos de experiência}

Como o objetivo dessa pesquisa é investigar as principais lições aprendidas por parte de empresas de desenvolvimento de software ao tentar adotar Ágil, além das publicações científicas, também foram considerados relatos de experiências provenientes de apresentações em conferências relevantes. Neste sentido, foi realizada uma revisão exploratória, detalhada na seção abaixo.

\subsection{Método de pesquisa}

Para a seleção de relatos de experiência foi feita uma pesquisa exploratória, método bem mais flexível se comparado à revisão sistemática. As fontes utilizadas nessa etapa foram as conferências de Engenharia de Software representativas internacionalmente, como: \textit{``Agile Conference",``Agile Brazil",``Agile Mumbai",``Agile Goa"}, etc.

Uma análise qualitativa foi efetuada em cada material encontrado, mantendo-se o critério de selecionar trabalhos recentes e que contribuissem com os objetivos da pesquisa.

\subsection{Relatos selecionados}

A Tabela \ref{tab:relatosEncontrados} apresenta informações a respeito dos relatos de experiência selecionados para a pesquisa (título, conferência e ano em que foram apresentados).

\begin{table}[H]
	\centering
	\captionsetup{justification=centering,margin=1cm}
	\begin{tabular}{| c | c | m{8cm} | m{2.5cm} |} \hline \textbf{Ano} & \textbf{Conferência}  & \textbf{Título do trabalho} & \textbf{Referência} \\ \hline
		\multirow{1}{*}{2008}
			& Agile Mumbai & Value-Driven Agile Adoption & \cite{Ahmed2008} \\ \cline{2-4}
		\hline \hline
		\multirow{4}{*}{2012}
			& \multirow{1}{*}{Agile Conference}
				& An Agile Adoption and Transformation Survival Guide & \cite{Sahota2012} \\ \cline{2-4}
			& \multirow{2}{*}{Agile Brazil}
				& Abolições Aprendidas: agilidade fora do contexto ideal & \cite{Piegas2012} \\ \cline{3-4}
				&& Implantando a Cultura Ágil em Larga Escala & \cite{Parzinello2012} \\ \cline{2-4}
			& \multirow{1}{*}{Agile Goa}
				& Agile in IT services & \cite{Srinath2012} \\ \cline{2-4}
		\hline \hline
		\multirow{18}{*}{2013}
			& \multirow{4}{*}{Agile Conference}
				& Evolving to Agile: transforming a public sector organization & \cite{Karaj2013} \\ \cline{3-4}
				&& Lean Change: Enabling Agile Transformation through Lean Startup, Kanban, and Kotter & \cite{Hui2013} \\ \cline{2-4}
			& \multirow{12}{*}{Agile Brazil}
				& 4 atitudes para melhorar a agilidade de uma empresa. Na prática. & \cite{Valerio2013} \\ \cline{3-4}
				&& Desafios do Desenvolvimento Ágil para o Governo & \cite{Stefano2013} \\ \cline{3-4}
				&& Padrões e Antipadrões da Adoção da Agilidade em Governo & \cite{Rodrigues2013} \\ \cline{3-4}
				&& Agilidade das Trincheiras do Tribunal Superior de Trabalho & \cite{Vieira2013} \\ \cline{3-4}
				&& Agile Black Ops - Como infiltrar agile em ambiente hostil & \cite{Queiroz2013} \\ \cline{3-4}
				&& Adoção de práticas ágeis no desenvolvimento de software de missão crítica & \cite{Bastos2013} \\ \cline{3-4}
				&& A incrível História de uma Organização Pública que Acredita em Agilidade & \cite{Maciel2013} \\ \cline{2-4}
		\hline
	\end{tabular}
	\caption{Relatos de experiência utilizados na pesquisa encontrados nas principais conferências relevantes}
	\label{tab:relatosEncontrados}
\end{table}

Dentre os trabalhos selecionados, a apresentação\cite{Ahmed2008} demonstrou um framework para auxiliar no processo de adoção ágil. O trabalho \cite{Sahota2012} criou um guia para não apenas se adotar Ágil, mas sim transformar a organização, fazer com que ela de fato seja ágil. O trabalho \cite{Piegas2012} listou dezessete lições aprendidas ao se adotar Ágil fora do contexto ideal. O relato \cite{Parzinello2012} compartilhou uma experiência sobre um processo de adesão ágil gradual em larga escala. O conteúdo de \cite{Srinath2012} relacionou desafios e lições aprendidas descobertas durante a utilização de Ágil em uma organização de serviços de TI. A apresentação \cite{Karaj2013} detalhou todo o processo de maturação e uso de metodologias ágeis em uma organização pública. Foram demonstrados em \cite{Hui2013} princípios de Lean Startup que auxiliam no processo de transformação ágil. No relato de experiência \cite{Valerio2013} apresentaram-se algumas práticas e atitudes tomadas em uma equipe em formação com o objetivo de alinhar e compartilhar o conhecimento entre todos, evitando as chamadas ilhas de conhecimento. O relato \cite{Stefano2013} compartilhou desafios enfrentados pela empresa Digitho Brasil ao adotar uma abordagem de desenvolvimento ágil para o governo. O trabalho \cite{Rodrigues2013} mostrou o caso de adoção ágil do VisPublica (projeto sobre visualização de dados públicos). A palestra \cite{Vieira2013}, apresentada no Agile Brazil 2013 em Brasília-DF, relatou os problemas e soluções encontrados pelo TST na implantação de métodos ágeis de desenvolvimento de software. A apresentação \cite{Queiroz2013}, utilizando uma abordagem problema/solução/resultado, compartilhou experiências sobre o uso de metodologias ágeis em ambientes hostis (não-ideais). O estudo \cite{Bastos2013} demonstrou o impacto causado por Ágil num projeto chamado Ecossistema de Urna, responsável pela automação de atividades e processos envolvendo a urna eletrônica no Brasil. O relato \cite{Maciel2013} descreveu a experiência do SERPRO Recife sobre sua busca pela agilidade.

\nomenclature{TST}{Tribunal Superior do Trabalho}%
