\chapter*{Abstract}

Agile development methodologies aim to build software in a faster and more efficient way, creating a minimum product that already brings value to the customer and, thenceforth, adding new features to it. This is a complete paradigm shift if we compare them with the planning-driven methodologies.

Changing drastically is never easy. Given this scenario, a research line focused on the study of this transition process from planning-driven methodologies to Agile has emerged. According to its studies, the main benefits from the adoption of agile methods are: customer satisfaction, more frequent deliveries, maintainability of the team morale high, improvement on the quality of the final product and so on. However, conquering all these benefits requires a lot of discipline and dedication.

Analyzing the results obtained with this research, a more common set of lessons learned was perceived. Among them, we can cite: ``customization and adaptability", ``experience, training and learning", ``engagement, commitment, discipline and teamwork" and ``technical and technological aspects".

\textbf{Keywords:} Agile methods, agile adoption, lessons learned
