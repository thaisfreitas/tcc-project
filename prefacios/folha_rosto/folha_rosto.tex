%
\hyphenation{}

\vspace*{0.0cm}
{\center
{\Large Thais Moura de Freitas}\\[2.4cm]
{\huge \bf Aplicação de Uma Estratégia de Automação Contínua de Testes no Desenvolvimento de Software}\\[2.0cm]
{\Large Orientador: Teresa Maciel}}\\[2.0cm]


{\raggedleft
\begin{minipage}[t]{8.3cm}
\setlength{\baselineskip}{0.25in}
Monografia apresentada ao Curso Bacharelado em Sistemas de Informação  da Universidade Federal Rural de Pernambuco, como requisito parcial para obtenção do título de Bacharel em Sistemas de Informação.\end{minipage}\\[2cm]}
\vspace{3cm}
{\center Recife \\[3mm]
Janeiro de 2015 \\}

\newpage
\vspace*{18cm}
{\raggedleft
\begin{minipage}[t]{6.0cm}
\setlength{\baselineskip}{0.25in}
Aos meus pais, Marcia e Abraão\\
Aos meus orientadores, Teresa e Giordano\\
Aos meus amigos\\
\end{minipage}\\[2cm]}

\newpage
\begin{center}
{\Large \bf Agradecimentos}
\end{center}
\vspace*{-0.06in}

Aos meus pais, por me darem condições de estudar e de mostrar sempre a importância do conhecimento, tanto na vida profissional quanto na pessoal. Sempre acreditando em mim e mostrando que eu poderia conseguir tudo aquilo que eu almejasse, me deixando ciente que todo esforço investido sempre é recompensado e que eu não devo desistir nunca de correr atrás dos meus sonhos. 

Aos meus orientadores Teresa Maciel e Giordano Cabral, que guiaram esse trabalho e conseguira alinhar minhas motivações pessoais com as requeridas computacionalmente para um trabalho de conclusão de curso. Em especial a Maria Conceição (Professora do Curso de Sistemas de informações - UFRPE) que foi minha orientadora de projeto de iniciação científica e da monitoria durante a graduação e que me deu várias oportunidades para construção do conhecimento. A todo conhecimento que a Universidade Federal Rural de Pernambuco e seus professores me proporcionaram. 

Aos meus amigos, que representam uma grande parte do meu incentivo, confiança e por acreditar que sou capaz de conquistar e ultrapassar barreiras. A Rebeka Maranhão (M.a em Ciências Biológicas - UFPE ), Julia Mariana (Graduanda em Administração - UPE)  por conseguirem me acalmar e dar conselhos especiais, por terem me encorajado com essa fase difícil e por estarem sempre ao meu lado. A tantos outros amigos que também participaram, apoiaram com conversas e com paciência para que toda essa jornada acabasse. A Johann Gomes (Bacharel em Sistemas de Informações na UFRPE) por ter sido um grande companheiro durante toda jornada na graduação e por partilhar as dificuldades, tornando esses desafios mais fáceis. A Luan Reis (Graduando em Sistemas de Informações na UFRPE) companheiro de projeto, por ser um grande amigo e dividir os momentos mais difíceis da nossa tragetória dentro da graduação e em especial no começo da nossa vida profissional, dividindo momentos difíceis e especiais. Agradeço especialmente, a Marilia Andrade (Médica Veterinária - UFRPE) por ter sido uma grande e especial companheira nesses momentos difíceis. A ThoughtWorks empresa que trabalho atualmente, aos amigos que fiz lá e ao meu time atual, por ter me fornecido todo suporte necessário para que conseguisse conciliar e finalizar este trabalho de conclusão de curso. 
